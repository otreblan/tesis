\documentclass[../main.tex]{subfiles}
\graphicspath{{\subfix{../images/}}}
\begin{document}

\customchapter{RESUMEN}

\introsection{Introducción a la Investigación}
Actualmente es tedioso y lento buscar datos en una colección de documentos escaneados.
En este trabajo crearemos una aplicación con Transformers para facilitar la búsqueda e indexación de esos datos.
Además la búsqueda será en lenguaje natural.

\introsection{Marco Metodológico}
Para el entrenamiento usamos un dataset de 500.000 imágenes sintéticas en español.
\# TODO Finetuning
% Aún no he solucionado el problema

\introsection{Resultados}
\#TODO

\introsection{Conclusiones}
\#TODO

%Un resumen debe ser un texto informativo, breve, claro, objetivo y veraz y debe permitir a los lectores entender el contenido del documento rápidamente. Un resumen es como una versión en miniatura de su trabajo y debe entenderse sin leer el documento. No contiene figuras, tablas, citas, referencias, abreviaturas y siglas. Se recomienda usar la estructura IMRyC:
%\begin{itemize}
%  \item \textbf{Introducción a la Investigación:} Sección que sintetiza qué es lo que se conocía sobre el problema de investigación, antecedentes y el propósito de la investigación. ¿Qué problema se intentó resolver? ¿Cuál era el objetivo de la investigación? ¿Por qué era importante este problema y los resultados de la investigación? \textit{(2-3 oraciones)}.
%  \item \textbf{Marco Metodológico:} Usualmente incluye caracterización de la investigación y la declaración del cómo se resolvió el problema. ¿Cómo procedió en resolver el problema?. No se describen métodos, ni instrumentos ni variables, pero la declaración debe ser suficientemente precisa para transmitir el diseño y el perfil de la investigación \textit{(3-4 oraciones)}.
%  \item \textbf{Resultados:} El texto debe ser descriptivo y suficientemente informativo de los resultados más significativos en el logro de los objetivos. ¿Cuál fue la respuesta? Exponga aquí los resultados en números y cantidades. Incluya en el resumen los hallazgos más importantes \textit{(4-6 oraciones)}.
%  \item \textbf{Conclusiones:} A través del mensaje breve y preciso, se transmite la interpretación final de los logros de la investigación y otros hallazgos importantes. ¿Cuáles son las implicaciones de sus respuestas? Una correcta redacción de esta parte motivará al lector a seguir leyendo \textit{(2-3 oraciones)}.
%\end{itemize}
%
%

\noindent \textbf{Palabras clave:}\\
\noindent LLM; OCR; Transformer;
\end{document}
