\documentclass[../main.tex]{subfiles}
\graphicspath{{\subfix{../images/}}}
\begin{document}

\chapter{MARCO METODOLÓGICO}

%Esta sección describe la propuesta metodológica. La metodología debe ser clara y debe tener un orden lógico. Ella debe tener relación con la problemática y los objetivos propuestos en la sub-sección Objetivos. Se debe tener en cuidado con los términos utilizados en esta sección. Ellos deberían haber sido descritos en el Marco Teórico. En esta parte también es usual colocar un diagrama o esquema que muestre el proceso que seguirán. Recuerde que todo lo expresado aquí debe tener el próposito de ser completamente reproducido por el lector.

\section{Creación del dataset}
\subsection{Dump de Wikipedia en español}

El primer paso para crear el dataset es descargar un dump de Wikipedia en español desde \url{https://dumps.wikimedia.org/eswiki/latest/eswiki-latest-pages-articles.xml.bz2}.

\subsection{Corpus}
Luego convertimos el dump en un corpus con el siguiente script:
% https://www.kdnuggets.com/2017/11/building-wikipedia-text-corpus-nlp.html
\inputminted[bgcolor=codeBack, tabsize=2]{python}{make_wiki_corpus.py}
\begin{minted}[bgcolor=codeBack, tabsize=2]{bash}
$ ./make_wiki_corpus.py eswiki-latest-pages-articles.xml.bz2 eswiki.txt
\end{minted}

El resultado es \texttt{eswiki.txt}, un archivo corpus.
El cual será un input de Synthdog, la misma herramienta descrita por \citet{kim2022ocrfree}, para la creación de imágenes sintéticas en español.

\subsection{Synthdog}

El último paso para generar las 500.000 imágenes es ejecutar los siguientes comandos en la carpeta \texttt{synthdog} del repositorio de Donut\cite{kim2022ocrfree}.

\begin{minted}[bgcolor=codeBack, tabsize=2]{bash}
$ sed "s/enwiki.txt/eswiki.txt/" config_en.yaml > config_es.yaml
$ mv eswiki.txt resources/corpus/
$ synthtiger \
	--output ./outputs/SynthDoG_es \
	--count 500000 \
	--worker $(nproc) \
	--verbose \
	template.py \
	SynthDoG \
	config_es.yaml
\end{minted}

\section{Pre-entrenamiento}

%Lorem ipsum dolor sit amet, consectetur adipiscing elit. Cras scelerisque quis augue malesuada porttitor. Phasellus tincidunt ipsum quis metus pellentesque commodo. Phasellus vitae metus arcu. Pellentesque scelerisque ac nulla laoreet mollis. Vivamus blandit, velit eget convallis eleifend, nibh risus congue sapien, tristique bibendum justo lorem eget arcu. Quisque mollis tristique tellus, sit amet hendrerit purus lacinia id. Fusce sed nunc arcu. Phasellus porttitor convallis ipsum, sit amet tempus nulla mattis a \cite{Reumann2012}.

%Sed eu orci feugiat, mollis nunc convallis, lobortis lorem. Suspendisse in lacinia magna, vel elementum tortor. Suspendisse hendrerit commodo justo, vitae posuere nisl eleifend at. Morbi vehicula risus ut diam tempus, ac dapibus erat euismod. Nunc quis elementum mi, et dictum nulla. Mauris rhoncus diam ac orci laoreet lobortis. Fusce et augue efficitur, dignissim dui ac, convallis est.


\end{document}
